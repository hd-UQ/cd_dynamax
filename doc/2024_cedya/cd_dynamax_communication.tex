\documentclass[communication]{CEDYA}
% Please, do not use any LaTeX macro definition

\begin{document}

%-----------------------------------------------------------------------------
% Indicate the CEDYA section for your communication: DS-ODE PDE NAS NLA OC-IP AM ME OTHER
%
% -  Partial Differential Equations (PDE)
% -  Dynamical systems - Ordinary Differential Equations (DS-ODE)
% -  Numerical Analysis and Simulation (NAS)
% -  Numerical Linear Algebra (NLA)
% -  Optimal Control - Inverse Problems (OC-IP)
% -  Applied Mathematics to Industry, Social Sciences and Biology (AM)
% -  Mathematical Education (ME)
% -  Other: Scientific Calculus, Approximation Theory, Discrete Mathematics... (OTHER)


\CEDYAsection{DS-ODE}

%-----------------------------------------------------------------------------

% \title{title of your talk}
\title{Data-assimilation meets automatic differentiation for identification of dynamical systems from irregularly-sampled, noisy data}
%-----------------------------------------------------------------------------

\begin{authors}
% \author{lastname}{firstname}{affiliation}{email}
% Use \author* to indicate the author that will be the speaker
	\author{Levine}{Matthew}{
	Eric and Wendy Schmidt Center \& 
	Broad Institute of MIT and Harvard}{levinema@broadinstitute.org }
\author*{Urteaga}{I\~nigo}{
	Basque Center for Applied Mathematics \&
	Ikerbasque, Basque Foundation
	for Science
}{iurteaga@bcamath.org}
\end{authors}

%-----------------------------------------------------------------------------

%To submit your proposal it is not necessary to be previously inscribed.
%However, if the speaker of an accepted communication is not inscribed before the end of May the communication will be cancelled.
%After your proposal is ready, you must send your (\LaTeX\ source) file by email to \texttt{cedya2024.commun@sema.org.es}. The deadline is 9th of April.

\begin{abstract}
% Describe the scientific content of the proposed talk
% The abstract should not exceed 1 page.

We present advances on leveraging automatic differentiation ---
computer-based evaluations,
via repeated application of the chain rule,
of partial derivatives of a software defined function
---
for learning and predicting
continuous (dynamics), discrete (observation) dynamical systems
that underpin real-world messy time-series data.

We study (possibly unknown) stochastic dynamical systems
$\dot{x} = f(x,t) + L(x,t) \dot{w}$ with $x(0)=x_0$, 
where $x \in \mathbb{R}^{d_x}$, $x_0 \sim \mathcal{N}(\mu_0, \Sigma_0)$, $f$ is a possibly time-dependent drift function, $L$ a possibly state and/or time-dependent diffusion coefficient, and $\dot{w}$ the derivative of a $d_x$-dimensional Brownian motion with a covariance $Q$.
Observed data are available at \emph{arbitrary times} ${t_k}_{k=1}^K$ collected via a \emph{noisy measurement} process 
$y(t) = h\big(x(t)\big) + \eta(t)$, where $h: \mathbb{R}^{d_x} \mapsto \mathbb{R}^{d_y}$ creates a $d_y$-dimensional observation from the $d_x$-dimensional state of the dynamical system $x(t)$ (a realization of the above SDE), with additive Gaussian noise $\eta(t)\sim \mathcal{N}(0, \Sigma_{\eta})$ applies additive Gaussian noise to the observation. We denote the collection of all parameters as $\theta = \{f, L, \mu_0, \Sigma_0, L, Q, h, \Sigma_{\eta}\}$.
%
Given a sequence of irregularly sampled, noisy observations $Y_K = [y(t_1), \dots , y(t_K)]$,
we wish to
($i$) filter ---estimate $p(x(t_K) | Y_K, \theta$),
($ii$) smooth ---estimate $p({x(t)}_t | Y_K, \theta$)
($iii$) predict ---estimate $p(x(t > t_K) | Y_K, \theta$),
and ($iv$) infer parameters (estimate $p(\theta | Y_K$),
for both linear and non-linear unknown functions $f$ and $h$.

We merge machine learning tools (i.e., automatic differentiation) 
with state-of-the-art data-assimilation
to solve all these interconnected Bayesian inference problems~\cite{sarkka},
%(note that parameter inference relies on marginalizing out unobserved states ${x(t)}_t$).
%This marginalization can be performed (approximately, in cases of non-linear dynamics) via filtering/smoothing algorithms. 
and devise a framework that allows for differentiation through filtering/smoothing algorithms and the SDE solver~\cite{cheng2022}.
%in partially-observed continuous-discrete linear and nonlinear dynamical systems.
%
By virtue of this novel synergy,
we enable usage of modern optimization and inference techniques
(e.g., stochastic gradient descent, Hamiltonian Monte Carlo)
for parameter inference and system identification of 
continuous-time dynamics with measurements sampled at possibly irregular discrete times.
Our work opens up novel research directions on
uncertainty quantification
and the combination of mechanistic and machine-learning models~\cite{matt}
for improved dynamical system identification from irregularly-sampled, noisy data.

%- If $\eta(t)$ had temporal correlations, we would likely adopt a mathematical setting that defines the observation process continuously in time via its own SDE.
%- Other extensions of the above paradigm include categorical state-spaces and non-additive observation noise distributions; these can fit into our code framework (indeed, they are covered in `dynamax`), but have not been our focus; thus, we specify our mathematical setting to distinguish from these cases.




\end{abstract}

%-----------------------------------------------------------------------------

\begin{thebibliography}{99}
%\bibitem{key} Info of the paper, book, ...
\bibitem{sarkka} Simo S\"arkk\"a and Lennart Svensson (2023).
Bayesian Filtering and Smoothing. Second
Edition. Cambridge University Press.
\bibitem{cheng2022} Yuming Chen, Daniel Sanz-Alonso, and Rebecca Willett. ``Autodifferentiable ensemble Kalman filters.'' SIAM Journal on Mathematics of Data Science 4, no. 2 (2022): 801-833.
\bibitem{matt} Matthew Levine and Andrew Stuart. ``A framework for machine learning of model error in dynamical systems.'' Communications of the American Mathematical Society 2, no. 07 (2022): 283-344.
\end{thebibliography}

%-----------------------------------------------------------------------------

\end{document}

